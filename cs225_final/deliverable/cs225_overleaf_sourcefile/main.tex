\documentclass[11pt]{amsart}
\usepackage{graphicx}
\usepackage{amssymb}
\usepackage{amsmath}
\usepackage{amsthm}
\usepackage{amsfonts}
\usepackage{bbm}
\usepackage{amsrefs}
\usepackage{tikz}
\usepackage{tikz-cd}
\usepackage{setspace,kantlipsum}
\usepackage[toc,page]{appendix}

\newtheorem{thm}[equation]{Theorem}
\newtheorem{lemma}[equation]{Lemma}
\newtheorem{col}[equation]{Corollary}

\theoremstyle{definition}
\newtheorem{defn}[equation]{Definition}
\newtheorem{eg}[equation]{Example}
\newtheorem*{note}{Note}

\usepackage[margin=2cm]{geometry}

\title{CS 225 Final Project Proposal}
\author{qc15, peilinh2, gzhai5, yikait2}
\date{April, 2021}

\begin{document}
\begin{spacing}{1}

\maketitle

% Your final project should have a clear conclusion or target goal – given a dataset and a code base that implements some graph algorithms, what can you learn from the dataset? Are you hoping to solve a specific problem? Are you hoping to produce a general search tool? Make sure that your entire team is on the same page for what a ‘successful’ project will look like. Be sure that your motivating question is solvable or your proposed final deliverable is reasonable as you will be expected to accomplish it.

\section{Leading Question}
    For our main goal, we want to solve the page rank problem. PageRank can be thought of as a model of user behavior. We assume there is a random surfer who is given a web page at random and keeps clicking on links, never hitting “back”, and he will keep doing this until he achieves a steady state. Thus given a graph, its nodes can be seen as websites, and its (directed) edges can be seen as links within the source webpage, and the weight on the edges determines probability of him clicking that link. The main problem for us is to determine the probability of the surfer browsing a given website after a period of time long enough to reach the steady state.
    
    In this case, we can modify the use of Page Rank to airlines and airports. With the nodes being airports, edges being airlines, and weight being the distance of taking that flight. With page rank, we can figure out the importance of airports in the world.
    
    As a side project, a graph can be seen as a CW complex as well as a simplex. Thus we can compute the homology, cohomology, as well as the fundamental group of the topological space. This goal can be achieved by a spanning tree algorithm plus a certain degree of math knowledge. We will treat this part of the project as the least important and might skip this completely depending on the amount of time left after we finish everything else.

\vspace{0.6cm}
\section{Dataset Acquisition and Processing}
    % Your final project must use at least one publicly accessible dataset and your proposal must clearly describe what dataset you have chosen to use. As part of this description, you must state how you will download, store, and process your data. This includes succinctly describing the data format and clearly identifying how said data will be used in your proposed graph data structures. Be advised that real world data is often messy and your proposed datasets may have errors or missing entries. Your proposal should also briefly describe how your group will handle these potential errors.
    For our final project, we are going to use the dataset of "OpenFlights". Open flights is an open-source data set of flight routes and airports suggested in the given documents. We will download the .dat files which includes the data of airlines, airport names, cities, and countries from the OpenFlight website. For processing those data, We plan to use airports as nodes, airlines as edges, and use distance as the weight. More specifically, we will use the airport data to compute the distance between each airport and use the reciprocal of those distances as the weight. 

    The website of the above mentioned dataset is ``https://openflights.org/data.html". The download links for the two dat files are ``https://raw.githubusercontent.com/jpatokal/openflights/master/data/airports-extended.dat" and ``https://raw.githubusercontent.com/jpatokal/openflights/master/data/airlines.dat".


\vspace{0.6cm}
\section{Graph Algorithms}
    The followings are the algorithms and goals we will try to implement throughout this month:
    
    \begin{itemize}
        \item Graph Traversal (including BFS, and possibly DFS).
            \begin{itemize}
                \item input: a graph (might have multiple path components).
                \item output: a traversal of the graph given.
                \item target big-O: $O(m+n)$, with $n$ is the number of Nodes and $m$ is the number of edges.
            \end{itemize}
        \item Minimal Spanning Tree (including Kruskul's Algorithm, and possibly Prim's).
            \begin{itemize}
                \item input: a graph (might have multiple path components).
                \item output: a minimal spanning tree of the graph given.
                \item target big-O: $O(n+m\cdot lg(n)).$
            \end{itemize}
            
            \begin{itemize}
                \item (possibly) Use spanning tree to calculate the homology groups of the graph (and possibly cohomology and fundamental group).
                \begin{itemize}
                    \item input: a graph (might have multiple path components).
                    \item output: the fundamental group, homology and cohomology groups of the graph.
                    \item target big-O: $O(n+m\cdot lg(n)).$
                \end{itemize}
            \end{itemize}
        \item Page Rank algorithm.
        \begin{itemize}
                \item input: a graph (might have multiple path components).
                \item output: the steady state of Page rank.
                \item target big-O: $O(n^2)*p$, where $p$ is the number of iterations to reach steady state.
            \end{itemize}
    \end{itemize}



\vspace{0.6cm}
\section{Timeline}
    Since we meet the group discussion every Monday evening, we are going to set the coming five Mondays (4/12, 4/19, 4/26, 5/3, 5/10) as the deadline for different segments of our final project. We plan to finish the graph traversal and minimal spanning tree at most in 2 weeks. (Mid-Project Checkin). For the third week, we are going to write the code for page rank. We hope we could end our project one week earlier than the final week so that we can still have another week to optimize our code and consider adding more extra features. 
\end{spacing}
\end{document}